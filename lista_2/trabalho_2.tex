% Options for packages loaded elsewhere
\PassOptionsToPackage{unicode}{hyperref}
\PassOptionsToPackage{hyphens}{url}
%
\documentclass[
]{article}
\title{Trabalho de R Modelos - FIAP}
\author{}
\date{\vspace{-2.5em}}

\usepackage{amsmath,amssymb}
\usepackage{lmodern}
\usepackage{iftex}
\ifPDFTeX
  \usepackage[T1]{fontenc}
  \usepackage[utf8]{inputenc}
  \usepackage{textcomp} % provide euro and other symbols
\else % if luatex or xetex
  \usepackage{unicode-math}
  \defaultfontfeatures{Scale=MatchLowercase}
  \defaultfontfeatures[\rmfamily]{Ligatures=TeX,Scale=1}
\fi
% Use upquote if available, for straight quotes in verbatim environments
\IfFileExists{upquote.sty}{\usepackage{upquote}}{}
\IfFileExists{microtype.sty}{% use microtype if available
  \usepackage[]{microtype}
  \UseMicrotypeSet[protrusion]{basicmath} % disable protrusion for tt fonts
}{}
\makeatletter
\@ifundefined{KOMAClassName}{% if non-KOMA class
  \IfFileExists{parskip.sty}{%
    \usepackage{parskip}
  }{% else
    \setlength{\parindent}{0pt}
    \setlength{\parskip}{6pt plus 2pt minus 1pt}}
}{% if KOMA class
  \KOMAoptions{parskip=half}}
\makeatother
\usepackage{xcolor}
\IfFileExists{xurl.sty}{\usepackage{xurl}}{} % add URL line breaks if available
\IfFileExists{bookmark.sty}{\usepackage{bookmark}}{\usepackage{hyperref}}
\hypersetup{
  pdftitle={Trabalho de R Modelos - FIAP},
  hidelinks,
  pdfcreator={LaTeX via pandoc}}
\urlstyle{same} % disable monospaced font for URLs
\usepackage[margin=1in]{geometry}
\usepackage{color}
\usepackage{fancyvrb}
\newcommand{\VerbBar}{|}
\newcommand{\VERB}{\Verb[commandchars=\\\{\}]}
\DefineVerbatimEnvironment{Highlighting}{Verbatim}{commandchars=\\\{\}}
% Add ',fontsize=\small' for more characters per line
\usepackage{framed}
\definecolor{shadecolor}{RGB}{248,248,248}
\newenvironment{Shaded}{\begin{snugshade}}{\end{snugshade}}
\newcommand{\AlertTok}[1]{\textcolor[rgb]{0.94,0.16,0.16}{#1}}
\newcommand{\AnnotationTok}[1]{\textcolor[rgb]{0.56,0.35,0.01}{\textbf{\textit{#1}}}}
\newcommand{\AttributeTok}[1]{\textcolor[rgb]{0.77,0.63,0.00}{#1}}
\newcommand{\BaseNTok}[1]{\textcolor[rgb]{0.00,0.00,0.81}{#1}}
\newcommand{\BuiltInTok}[1]{#1}
\newcommand{\CharTok}[1]{\textcolor[rgb]{0.31,0.60,0.02}{#1}}
\newcommand{\CommentTok}[1]{\textcolor[rgb]{0.56,0.35,0.01}{\textit{#1}}}
\newcommand{\CommentVarTok}[1]{\textcolor[rgb]{0.56,0.35,0.01}{\textbf{\textit{#1}}}}
\newcommand{\ConstantTok}[1]{\textcolor[rgb]{0.00,0.00,0.00}{#1}}
\newcommand{\ControlFlowTok}[1]{\textcolor[rgb]{0.13,0.29,0.53}{\textbf{#1}}}
\newcommand{\DataTypeTok}[1]{\textcolor[rgb]{0.13,0.29,0.53}{#1}}
\newcommand{\DecValTok}[1]{\textcolor[rgb]{0.00,0.00,0.81}{#1}}
\newcommand{\DocumentationTok}[1]{\textcolor[rgb]{0.56,0.35,0.01}{\textbf{\textit{#1}}}}
\newcommand{\ErrorTok}[1]{\textcolor[rgb]{0.64,0.00,0.00}{\textbf{#1}}}
\newcommand{\ExtensionTok}[1]{#1}
\newcommand{\FloatTok}[1]{\textcolor[rgb]{0.00,0.00,0.81}{#1}}
\newcommand{\FunctionTok}[1]{\textcolor[rgb]{0.00,0.00,0.00}{#1}}
\newcommand{\ImportTok}[1]{#1}
\newcommand{\InformationTok}[1]{\textcolor[rgb]{0.56,0.35,0.01}{\textbf{\textit{#1}}}}
\newcommand{\KeywordTok}[1]{\textcolor[rgb]{0.13,0.29,0.53}{\textbf{#1}}}
\newcommand{\NormalTok}[1]{#1}
\newcommand{\OperatorTok}[1]{\textcolor[rgb]{0.81,0.36,0.00}{\textbf{#1}}}
\newcommand{\OtherTok}[1]{\textcolor[rgb]{0.56,0.35,0.01}{#1}}
\newcommand{\PreprocessorTok}[1]{\textcolor[rgb]{0.56,0.35,0.01}{\textit{#1}}}
\newcommand{\RegionMarkerTok}[1]{#1}
\newcommand{\SpecialCharTok}[1]{\textcolor[rgb]{0.00,0.00,0.00}{#1}}
\newcommand{\SpecialStringTok}[1]{\textcolor[rgb]{0.31,0.60,0.02}{#1}}
\newcommand{\StringTok}[1]{\textcolor[rgb]{0.31,0.60,0.02}{#1}}
\newcommand{\VariableTok}[1]{\textcolor[rgb]{0.00,0.00,0.00}{#1}}
\newcommand{\VerbatimStringTok}[1]{\textcolor[rgb]{0.31,0.60,0.02}{#1}}
\newcommand{\WarningTok}[1]{\textcolor[rgb]{0.56,0.35,0.01}{\textbf{\textit{#1}}}}
\usepackage{graphicx}
\makeatletter
\def\maxwidth{\ifdim\Gin@nat@width>\linewidth\linewidth\else\Gin@nat@width\fi}
\def\maxheight{\ifdim\Gin@nat@height>\textheight\textheight\else\Gin@nat@height\fi}
\makeatother
% Scale images if necessary, so that they will not overflow the page
% margins by default, and it is still possible to overwrite the defaults
% using explicit options in \includegraphics[width, height, ...]{}
\setkeys{Gin}{width=\maxwidth,height=\maxheight,keepaspectratio}
% Set default figure placement to htbp
\makeatletter
\def\fps@figure{htbp}
\makeatother
\setlength{\emergencystretch}{3em} % prevent overfull lines
\providecommand{\tightlist}{%
  \setlength{\itemsep}{0pt}\setlength{\parskip}{0pt}}
\setcounter{secnumdepth}{-\maxdimen} % remove section numbering
\ifLuaTeX
  \usepackage{selnolig}  % disable illegal ligatures
\fi

\begin{document}
\maketitle

Carregando as bibliotecas

\begin{Shaded}
\begin{Highlighting}[]
\CommentTok{\#install.packages("ggcorrplot")}
\CommentTok{\#install.packages("rattle")}

\FunctionTok{library}\NormalTok{(tidyverse)}
\end{Highlighting}
\end{Shaded}

\begin{verbatim}
## -- Attaching packages --------------------------------------- tidyverse 1.3.1 --
\end{verbatim}

\begin{verbatim}
## v ggplot2 3.3.5     v purrr   0.3.4
## v tibble  3.1.5     v dplyr   1.0.7
## v tidyr   1.1.4     v stringr 1.4.0
## v readr   2.0.2     v forcats 0.5.1
\end{verbatim}

\begin{verbatim}
## -- Conflicts ------------------------------------------ tidyverse_conflicts() --
## x dplyr::filter() masks stats::filter()
## x dplyr::lag()    masks stats::lag()
\end{verbatim}

\begin{Shaded}
\begin{Highlighting}[]
\FunctionTok{library}\NormalTok{(readr)}
\FunctionTok{library}\NormalTok{(ggcorrplot)}
\end{Highlighting}
\end{Shaded}

Carregando os dados

\begin{Shaded}
\begin{Highlighting}[]
\NormalTok{ds }\OtherTok{\textless{}{-}} \FunctionTok{read\_delim}\NormalTok{(}\StringTok{"consolidado\_para\_analise.csv"}\NormalTok{, }
    \AttributeTok{delim =} \StringTok{";"}\NormalTok{, }\AttributeTok{escape\_double =} \ConstantTok{FALSE}\NormalTok{, }\AttributeTok{trim\_ws =} \ConstantTok{TRUE}\NormalTok{, }\AttributeTok{show\_col\_types =} \ConstantTok{FALSE}\NormalTok{)}
\end{Highlighting}
\end{Shaded}

\begin{itemize}
\tightlist
\item
  Analisando quantidade de observações e variáveis
\end{itemize}

\begin{Shaded}
\begin{Highlighting}[]
\FunctionTok{dim}\NormalTok{(ds)}
\end{Highlighting}
\end{Shaded}

\begin{verbatim}
## [1] 473  18
\end{verbatim}

Conhecendo as variáveis presentes neste dataset

\begin{Shaded}
\begin{Highlighting}[]
\FunctionTok{names}\NormalTok{(ds)}
\end{Highlighting}
\end{Shaded}

\begin{verbatim}
##  [1] "ID"                    "DataNascimento"        "Sexo"                 
##  [4] "TempodeServiço"        "EstadoCivil"           "NumerodeFilhos"       
##  [7] "TempodeResidencia"     "Conta"                 "salario"              
## [10] "data_atual"            "faixa_salario"         "default"              
## [13] "default1"              "QtdaParcelas"          "Atraso"               
## [16] "ValorEmprestimo"       "QtdaPagas"             "comprometido_de_renda"
\end{verbatim}

\begin{itemize}
\tightlist
\item
  Anotações importantes para análise dos dados:
\end{itemize}

\begin{Shaded}
\begin{Highlighting}[]
\CommentTok{\# ID {-}\textgreater{} Identificador}

\CommentTok{\# Qualitativas:}
\CommentTok{\# Sexo {-}\textgreater{} nominal (Feminino, Masculino) }
\CommentTok{\# EstadoCivil {-}\textgreater{} nominal (1,2,3,4) }
\CommentTok{\# Conta {-}\textgreater{} nominal (empresa, Particular) }
\CommentTok{\# faixa\_salario {-}\textgreater{} nominal (A,B,C,D) }
\CommentTok{\# Atraso {-}\textgreater{} nominal (Sim, Não) }

\CommentTok{\# Quantitativas:}
\CommentTok{\# NumerodeFilhos {-}\textgreater{} discreta}
\CommentTok{\# TempodeServiço {-}\textgreater{} discreta}
\CommentTok{\# TempodeResidencia {-}\textgreater{} discreta}
\CommentTok{\# salario {-}\textgreater{} contínua}
\CommentTok{\# QtdaParcelas {-}\textgreater{} discreta}
\CommentTok{\# ValorEmprestimo {-}\textgreater{} contínua}
\CommentTok{\# QtdaPagas {-}\textgreater{} discreta}
\CommentTok{\# comprometido\_de\_renda {-}\textgreater{} contínua}

\CommentTok{\# Excluir:}
\CommentTok{\# data\_atual}
\CommentTok{\# ID}
\CommentTok{\# DataNascimento}

\CommentTok{\# Preditoras:}
\CommentTok{\# default {-}\textgreater{} Classificadora}
\CommentTok{\# default1 {-}\textgreater{} Regressora}
\end{Highlighting}
\end{Shaded}

\begin{itemize}
\tightlist
\item
  Criando um novo dataset para retirar algumas colunas
\end{itemize}

\begin{Shaded}
\begin{Highlighting}[]
\NormalTok{ds\_analise }\OtherTok{\textless{}{-}} \FunctionTok{select}\NormalTok{(ds, }\SpecialCharTok{{-}}\NormalTok{(ID), }\SpecialCharTok{{-}}\NormalTok{(data\_atual), }\SpecialCharTok{{-}}\NormalTok{(DataNascimento))}
\NormalTok{ds\_analise}
\end{Highlighting}
\end{Shaded}

\begin{verbatim}
## # A tibble: 473 x 15
##    Sexo      TempodeServiço EstadoCivil NumerodeFilhos TempodeResidencia Conta  
##    <chr>              <dbl>       <dbl>          <dbl>             <dbl> <chr>  
##  1 Masculino             98           3              0                 7 Partic~
##  2 Masculino             98           3              0                43 Partic~
##  3 Feminino              98           1              0                 7 empresa
##  4 Feminino              98           1              0                13 empresa
##  5 Masculino             98           2              3                 8 Partic~
##  6 Masculino             98           2              0                26 Partic~
##  7 Masculino             98           3              0                40 Partic~
##  8 Feminino              98           3              0                15 empresa
##  9 Feminino              98           2              6                34 empresa
## 10 Feminino              98           1              0                 0 empresa
## # ... with 463 more rows, and 9 more variables: salario <dbl>,
## #   faixa_salario <chr>, default <chr>, default1 <dbl>, QtdaParcelas <dbl>,
## #   Atraso <chr>, ValorEmprestimo <dbl>, QtdaPagas <dbl>,
## #   comprometido_de_renda <chr>
\end{verbatim}

\begin{itemize}
\tightlist
\item
  Verificando as variavéis que ficou no dataset
\end{itemize}

\begin{Shaded}
\begin{Highlighting}[]
\FunctionTok{names}\NormalTok{(ds\_analise)}
\end{Highlighting}
\end{Shaded}

\begin{verbatim}
##  [1] "Sexo"                  "TempodeServiço"        "EstadoCivil"          
##  [4] "NumerodeFilhos"        "TempodeResidencia"     "Conta"                
##  [7] "salario"               "faixa_salario"         "default"              
## [10] "default1"              "QtdaParcelas"          "Atraso"               
## [13] "ValorEmprestimo"       "QtdaPagas"             "comprometido_de_renda"
\end{verbatim}

\begin{itemize}
\tightlist
\item
  Verificando a correlação das variáveis númericas
\end{itemize}

\begin{Shaded}
\begin{Highlighting}[]
\CommentTok{\# NumerodeFilhos {-}\textgreater{} discreta}
\CommentTok{\# TempodeServiço {-}\textgreater{} discreta}
\CommentTok{\# TempodeResidencia {-}\textgreater{} discreta}
\CommentTok{\# salario {-}\textgreater{} contínua}
\CommentTok{\# QtdaParcelas {-}\textgreater{} discreta}
\CommentTok{\# ValorEmprestimo {-}\textgreater{} contínua}
\CommentTok{\# QtdaPagas {-}\textgreater{} discreta}
\CommentTok{\# comprometido\_de\_renda {-}\textgreater{} contínua}

\NormalTok{ds\_numericos }\OtherTok{\textless{}{-}} \FunctionTok{select\_if}\NormalTok{(ds\_analise, is.numeric)}
\NormalTok{correl }\OtherTok{\textless{}{-}}\FunctionTok{cor}\NormalTok{(ds\_numericos)}
\FunctionTok{ggcorrplot}\NormalTok{(correl)}
\end{Highlighting}
\end{Shaded}

\includegraphics{trabalho_2_files/figure-latex/unnamed-chunk-8-1.pdf}

\begin{Shaded}
\begin{Highlighting}[]
\CommentTok{\# As variáveis {-}\textgreater{} QtdaPagas, ValorEmprestimo e QtdaParcelas estão muito correlacionadas,}
\CommentTok{\# irei verificar como o modelo se comporta em relação a elas, e então irei decidir se tiro alguma}
\CommentTok{\# menos importante.}
\end{Highlighting}
\end{Shaded}

\begin{enumerate}
\def\labelenumi{\Alph{enumi})}
\tightlist
\item
  Criar o modelo de regressão logística com a variável target DEFAULT1 e
  interpretar os coeficientes e verificar o quanto o modelo teve de
  acurácia.
\end{enumerate}

\begin{Shaded}
\begin{Highlighting}[]
\NormalTok{logistica }\OtherTok{\textless{}{-}} \FunctionTok{glm}\NormalTok{(default1 }\SpecialCharTok{\textasciitilde{}}\NormalTok{ Sexo }\SpecialCharTok{+}\NormalTok{ EstadoCivil }\SpecialCharTok{+}\NormalTok{ NumerodeFilhos }\SpecialCharTok{+}\NormalTok{ TempodeResidencia }\SpecialCharTok{+}
\NormalTok{                    faixa\_salario }\SpecialCharTok{+}\NormalTok{ QtdaParcelas }\SpecialCharTok{+}\NormalTok{ Atraso }\SpecialCharTok{+}\NormalTok{ ValorEmprestimo }\SpecialCharTok{+}
\NormalTok{                   QtdaPagas }\SpecialCharTok{+}\NormalTok{ Conta }\SpecialCharTok{+}\NormalTok{ salario, }\AttributeTok{family =}\NormalTok{ binomial, }\AttributeTok{data =}\NormalTok{ ds\_analise)}
\FunctionTok{summary}\NormalTok{(logistica)}
\end{Highlighting}
\end{Shaded}

\begin{verbatim}
## 
## Call:
## glm(formula = default1 ~ Sexo + EstadoCivil + NumerodeFilhos + 
##     TempodeResidencia + faixa_salario + QtdaParcelas + Atraso + 
##     ValorEmprestimo + QtdaPagas + Conta + salario, family = binomial, 
##     data = ds_analise)
## 
## Deviance Residuals: 
##     Min       1Q   Median       3Q      Max  
## -2.1001  -1.0018   0.6065   0.7930   1.4440  
## 
## Coefficients: (1 not defined because of singularities)
##                     Estimate Std. Error z value Pr(>|z|)    
## (Intercept)        2.676e+00  7.680e-01   3.484 0.000493 ***
## SexoMasculino     -1.957e-01  2.598e-01  -0.753 0.451376    
## EstadoCivil        1.368e-02  1.095e-01   0.125 0.900562    
## NumerodeFilhos     1.481e-02  6.101e-02   0.243 0.808170    
## TempodeResidencia  1.117e-02  1.145e-02   0.975 0.329453    
## faixa_salarioB    -5.794e-01  3.309e-01  -1.751 0.079967 .  
## faixa_salarioC    -3.540e-02  7.756e-01  -0.046 0.963597    
## faixa_salarioD    -7.988e-01  1.285e+00  -0.622 0.534101    
## QtdaParcelas      -2.847e+00  9.141e+01  -0.031 0.975154    
## AtrasoSim          5.579e-01  3.290e-01   1.696 0.089876 .  
## ValorEmprestimo    6.689e-02  2.612e+00   0.026 0.979568    
## QtdaPagas          1.339e+00  3.990e-01   3.355 0.000792 ***
## ContaParticular           NA         NA      NA       NA    
## salario           -7.637e-06  1.923e-05  -0.397 0.691232    
## ---
## Signif. codes:  0 '***' 0.001 '**' 0.01 '*' 0.05 '.' 0.1 ' ' 1
## 
## (Dispersion parameter for binomial family taken to be 1)
## 
##     Null deviance: 556.26  on 472  degrees of freedom
## Residual deviance: 511.20  on 460  degrees of freedom
## AIC: 537.2
## 
## Number of Fisher Scoring iterations: 12
\end{verbatim}

\begin{Shaded}
\begin{Highlighting}[]
\CommentTok{\# * Através desta  primeira analise, a variável QtdaPagas apresentou mais significante que as demais}
\CommentTok{\# ValorEmprestimo e QtdaParcelas, ainda analisando o caso de correlação}
\CommentTok{\# Conta apresentou valores NA (1 not defined because of singularities), já vou excluir}
\end{Highlighting}
\end{Shaded}

\begin{itemize}
\tightlist
\item
  Retirando a variável QtdaParcelas, pois creio que o valoremprestimo
  seja mais relevante
\end{itemize}

\begin{Shaded}
\begin{Highlighting}[]
\NormalTok{logistica }\OtherTok{\textless{}{-}} \FunctionTok{glm}\NormalTok{(default1 }\SpecialCharTok{\textasciitilde{}}\NormalTok{ Sexo }\SpecialCharTok{+} 
\NormalTok{                            EstadoCivil }\SpecialCharTok{+} 
\NormalTok{                            NumerodeFilhos }\SpecialCharTok{+} 
\NormalTok{                            TempodeResidencia }\SpecialCharTok{+}
\NormalTok{                            faixa\_salario  }\SpecialCharTok{+} 
\NormalTok{                            Atraso }\SpecialCharTok{+} 
\NormalTok{                            ValorEmprestimo }\SpecialCharTok{+}
\NormalTok{                            QtdaPagas}\SpecialCharTok{+} 
\NormalTok{                            salario, }\AttributeTok{family =}\NormalTok{ binomial, }\AttributeTok{data =}\NormalTok{ ds\_analise)}
\FunctionTok{summary}\NormalTok{(logistica)}
\end{Highlighting}
\end{Shaded}

\begin{verbatim}
## 
## Call:
## glm(formula = default1 ~ Sexo + EstadoCivil + NumerodeFilhos + 
##     TempodeResidencia + faixa_salario + Atraso + ValorEmprestimo + 
##     QtdaPagas + salario, family = binomial, data = ds_analise)
## 
## Deviance Residuals: 
##     Min       1Q   Median       3Q      Max  
## -2.2220  -1.0240   0.6251   0.7935   1.4789  
## 
## Coefficients:
##                     Estimate Std. Error z value Pr(>|z|)    
## (Intercept)        2.822e+00  7.643e-01   3.692 0.000222 ***
## SexoMasculino     -1.940e-01  2.584e-01  -0.751 0.452765    
## EstadoCivil        3.534e-03  1.086e-01   0.033 0.974033    
## NumerodeFilhos     2.233e-02  6.073e-02   0.368 0.713074    
## TempodeResidencia  7.504e-03  1.122e-02   0.669 0.503680    
## faixa_salarioB    -5.852e-01  3.292e-01  -1.778 0.075438 .  
## faixa_salarioC     5.033e-02  7.723e-01   0.065 0.948035    
## faixa_salarioD    -6.919e-01  1.277e+00  -0.542 0.587986    
## AtrasoSim          5.607e-01  3.282e-01   1.709 0.087504 .  
## ValorEmprestimo   -1.186e-02  3.704e-03  -3.203 0.001362 ** 
## QtdaPagas          1.058e+00  3.874e-01   2.731 0.006313 ** 
## salario           -9.071e-06  1.916e-05  -0.474 0.635829    
## ---
## Signif. codes:  0 '***' 0.001 '**' 0.01 '*' 0.05 '.' 0.1 ' ' 1
## 
## (Dispersion parameter for binomial family taken to be 1)
## 
##     Null deviance: 556.26  on 472  degrees of freedom
## Residual deviance: 516.81  on 461  degrees of freedom
## AIC: 540.81
## 
## Number of Fisher Scoring iterations: 4
\end{verbatim}

\begin{itemize}
\tightlist
\item
  Retirando a variável faixa\_salario pois já tenho a variável de
  salário
\end{itemize}

\begin{Shaded}
\begin{Highlighting}[]
\NormalTok{logistica }\OtherTok{\textless{}{-}} \FunctionTok{glm}\NormalTok{(default1 }\SpecialCharTok{\textasciitilde{}}\NormalTok{ Sexo }\SpecialCharTok{+} 
\NormalTok{                            EstadoCivil }\SpecialCharTok{+} 
\NormalTok{                            NumerodeFilhos }\SpecialCharTok{+} 
\NormalTok{                            TempodeResidencia }\SpecialCharTok{+}
\NormalTok{                            Atraso }\SpecialCharTok{+} 
\NormalTok{                            ValorEmprestimo }\SpecialCharTok{+}
\NormalTok{                            QtdaPagas}\SpecialCharTok{+} 
\NormalTok{                            salario, }\AttributeTok{family =}\NormalTok{ binomial, }\AttributeTok{data =}\NormalTok{ ds\_analise)}
\FunctionTok{summary}\NormalTok{(logistica)}
\end{Highlighting}
\end{Shaded}

\begin{verbatim}
## 
## Call:
## glm(formula = default1 ~ Sexo + EstadoCivil + NumerodeFilhos + 
##     TempodeResidencia + Atraso + ValorEmprestimo + QtdaPagas + 
##     salario, family = binomial, data = ds_analise)
## 
## Deviance Residuals: 
##     Min       1Q   Median       3Q      Max  
## -2.3714  -1.0880   0.6518   0.8204   1.4254  
## 
## Coefficients:
##                     Estimate Std. Error z value Pr(>|z|)    
## (Intercept)        2.853e+00  6.382e-01   4.471 7.79e-06 ***
## SexoMasculino     -3.315e-01  2.446e-01  -1.355 0.175465    
## EstadoCivil        8.797e-04  1.072e-01   0.008 0.993450    
## NumerodeFilhos     1.203e-02  5.952e-02   0.202 0.839786    
## TempodeResidencia  8.044e-03  1.109e-02   0.725 0.468148    
## AtrasoSim          5.681e-01  3.252e-01   1.747 0.080694 .  
## ValorEmprestimo   -1.220e-02  3.690e-03  -3.306 0.000946 ***
## QtdaPagas          1.093e+00  3.860e-01   2.831 0.004643 ** 
## salario           -1.369e-05  6.589e-06  -2.077 0.037809 *  
## ---
## Signif. codes:  0 '***' 0.001 '**' 0.01 '*' 0.05 '.' 0.1 ' ' 1
## 
## (Dispersion parameter for binomial family taken to be 1)
## 
##     Null deviance: 556.26  on 472  degrees of freedom
## Residual deviance: 523.18  on 464  degrees of freedom
## AIC: 541.18
## 
## Number of Fisher Scoring iterations: 4
\end{verbatim}

\begin{itemize}
\tightlist
\item
  Retirando TempodeResidencia
\end{itemize}

\begin{Shaded}
\begin{Highlighting}[]
\NormalTok{logistica }\OtherTok{\textless{}{-}} \FunctionTok{glm}\NormalTok{(default1 }\SpecialCharTok{\textasciitilde{}}\NormalTok{ Sexo }\SpecialCharTok{+} 
\NormalTok{                            EstadoCivil }\SpecialCharTok{+} 
\NormalTok{                            NumerodeFilhos }\SpecialCharTok{+} 
\NormalTok{                            Atraso }\SpecialCharTok{+} 
\NormalTok{                            ValorEmprestimo }\SpecialCharTok{+}
\NormalTok{                            QtdaPagas}\SpecialCharTok{+} 
\NormalTok{                            salario, }\AttributeTok{family =}\NormalTok{ binomial, }\AttributeTok{data =}\NormalTok{ ds\_analise)}
\FunctionTok{summary}\NormalTok{(logistica)}
\end{Highlighting}
\end{Shaded}

\begin{verbatim}
## 
## Call:
## glm(formula = default1 ~ Sexo + EstadoCivil + NumerodeFilhos + 
##     Atraso + ValorEmprestimo + QtdaPagas + salario, family = binomial, 
##     data = ds_analise)
## 
## Deviance Residuals: 
##     Min       1Q   Median       3Q      Max  
## -2.2789  -1.0952   0.6626   0.8152   1.4533  
## 
## Coefficients:
##                   Estimate Std. Error z value Pr(>|z|)    
## (Intercept)      2.944e+00  6.266e-01   4.698 2.62e-06 ***
## SexoMasculino   -3.194e-01  2.440e-01  -1.309 0.190634    
## EstadoCivil     -2.905e-03  1.070e-01  -0.027 0.978342    
## NumerodeFilhos   9.428e-03  5.934e-02   0.159 0.873751    
## AtrasoSim        5.732e-01  3.249e-01   1.764 0.077726 .  
## ValorEmprestimo -1.228e-02  3.678e-03  -3.337 0.000846 ***
## QtdaPagas        1.101e+00  3.845e-01   2.864 0.004178 ** 
## salario         -1.369e-05  6.591e-06  -2.077 0.037792 *  
## ---
## Signif. codes:  0 '***' 0.001 '**' 0.01 '*' 0.05 '.' 0.1 ' ' 1
## 
## (Dispersion parameter for binomial family taken to be 1)
## 
##     Null deviance: 556.26  on 472  degrees of freedom
## Residual deviance: 523.72  on 465  degrees of freedom
## AIC: 539.72
## 
## Number of Fisher Scoring iterations: 4
\end{verbatim}

\begin{Shaded}
\begin{Highlighting}[]
\CommentTok{\# Obtive um ganho no modelo e também não a vejo como significante no contexto de negócio}
\end{Highlighting}
\end{Shaded}

\begin{itemize}
\tightlist
\item
  Retirando Atraso
\end{itemize}

\begin{Shaded}
\begin{Highlighting}[]
\NormalTok{logistica }\OtherTok{\textless{}{-}} \FunctionTok{glm}\NormalTok{(default1 }\SpecialCharTok{\textasciitilde{}}\NormalTok{ Sexo }\SpecialCharTok{+} 
\NormalTok{                            EstadoCivil }\SpecialCharTok{+} 
\NormalTok{                            NumerodeFilhos }\SpecialCharTok{+} 
\NormalTok{                            ValorEmprestimo }\SpecialCharTok{+}
\NormalTok{                            QtdaPagas}\SpecialCharTok{+} 
\NormalTok{                            salario, }\AttributeTok{family =}\NormalTok{ binomial, }\AttributeTok{data =}\NormalTok{ ds\_analise)}
\FunctionTok{summary}\NormalTok{(logistica)}
\end{Highlighting}
\end{Shaded}

\begin{verbatim}
## 
## Call:
## glm(formula = default1 ~ Sexo + EstadoCivil + NumerodeFilhos + 
##     ValorEmprestimo + QtdaPagas + salario, family = binomial, 
##     data = ds_analise)
## 
## Deviance Residuals: 
##     Min       1Q   Median       3Q      Max  
## -2.2841  -1.1088   0.6604   0.8189   1.4622  
## 
## Coefficients:
##                   Estimate Std. Error z value Pr(>|z|)    
## (Intercept)      3.220e+00  6.113e-01   5.268 1.38e-07 ***
## SexoMasculino   -2.873e-01  2.429e-01  -1.183 0.236922    
## EstadoCivil     -1.260e-02  1.065e-01  -0.118 0.905854    
## NumerodeFilhos   3.570e-03  5.917e-02   0.060 0.951896    
## ValorEmprestimo -1.221e-02  3.670e-03  -3.327 0.000879 ***
## QtdaPagas        1.075e+00  3.834e-01   2.805 0.005027 ** 
## salario         -1.478e-05  6.549e-06  -2.258 0.023963 *  
## ---
## Signif. codes:  0 '***' 0.001 '**' 0.01 '*' 0.05 '.' 0.1 ' ' 1
## 
## (Dispersion parameter for binomial family taken to be 1)
## 
##     Null deviance: 556.26  on 472  degrees of freedom
## Residual deviance: 527.07  on 466  degrees of freedom
## AIC: 541.07
## 
## Number of Fisher Scoring iterations: 4
\end{verbatim}

\begin{Shaded}
\begin{Highlighting}[]
\CommentTok{\# Obtive um ganho no modelo}
\end{Highlighting}
\end{Shaded}

\begin{itemize}
\tightlist
\item
  Vou realizar a acuracia do modelo
\end{itemize}

\begin{Shaded}
\begin{Highlighting}[]
\NormalTok{prob }\OtherTok{\textless{}{-}} \FunctionTok{predict}\NormalTok{(logistica, ds\_analise, }\AttributeTok{type =} \StringTok{\textquotesingle{}response\textquotesingle{}}\NormalTok{)}
\NormalTok{resultado }\OtherTok{\textless{}{-}} \FunctionTok{if\_else}\NormalTok{(prob }\SpecialCharTok{\textgreater{}=} \FloatTok{0.50}\NormalTok{, }\DecValTok{1}\NormalTok{,}\DecValTok{0}\NormalTok{)}
\NormalTok{target }\OtherTok{\textless{}{-}}\NormalTok{ ds\_analise}\SpecialCharTok{$}\NormalTok{default1 }
\NormalTok{tabela }\OtherTok{\textless{}{-}} \FunctionTok{table}\NormalTok{(target, resultado)}
\NormalTok{acuracia }\OtherTok{\textless{}{-}}\NormalTok{ ((tabela[}\DecValTok{1}\NormalTok{] }\SpecialCharTok{+}\NormalTok{ tabela[}\DecValTok{4}\NormalTok{]) }\SpecialCharTok{/} \FunctionTok{sum}\NormalTok{(tabela))}
\NormalTok{acuracia}
\end{Highlighting}
\end{Shaded}

\begin{verbatim}
## [1] 0.744186
\end{verbatim}

\begin{Shaded}
\begin{Highlighting}[]
\CommentTok{\# Acuracia de 74,41\%}
\end{Highlighting}
\end{Shaded}

\begin{enumerate}
\def\labelenumi{\alph{enumi})}
\setcounter{enumi}{1}
\tightlist
\item
  Criar um modelo de árvore de decisão com a variável target default e
  interpretar as regras do modelo e verificar o quanto o modelo teve de
  acurácia.
\end{enumerate}

\begin{Shaded}
\begin{Highlighting}[]
\FunctionTok{library}\NormalTok{(rpart)}
\FunctionTok{library}\NormalTok{(rattle)}
\end{Highlighting}
\end{Shaded}

\begin{verbatim}
## Warning: package 'rattle' was built under R version 4.1.2
\end{verbatim}

\begin{verbatim}
## Carregando pacotes exigidos: bitops
\end{verbatim}

\begin{verbatim}
## Rattle: A free graphical interface for data science with R.
## Version 5.4.0 Copyright (c) 2006-2020 Togaware Pty Ltd.
## Type 'rattle()' to shake, rattle, and roll your data.
\end{verbatim}

\begin{Shaded}
\begin{Highlighting}[]
\NormalTok{mytree }\OtherTok{=} \FunctionTok{rpart}\NormalTok{(default }\SpecialCharTok{\textasciitilde{}}\NormalTok{ Sexo }\SpecialCharTok{+} 
\NormalTok{                EstadoCivil }\SpecialCharTok{+} 
\NormalTok{                NumerodeFilhos }\SpecialCharTok{+} 
\NormalTok{                ValorEmprestimo }\SpecialCharTok{+}
\NormalTok{                QtdaPagas}\SpecialCharTok{+} 
\NormalTok{                salario,}
                \AttributeTok{data =}\NormalTok{ ds\_analise,}
                \AttributeTok{method=}\StringTok{"class"}\NormalTok{)}
\NormalTok{mytree}
\end{Highlighting}
\end{Shaded}

\begin{verbatim}
## n= 473 
## 
## node), split, n, loss, yval, (yprob)
##       * denotes terminal node
## 
## 1) root 473 130 Inadimplente (0.2748414 0.7251586)  
##   2) ValorEmprestimo>=947.5 64  28 Adimplente (0.5625000 0.4375000)  
##     4) salario>=27825 35  11 Adimplente (0.6857143 0.3142857) *
##     5) salario< 27825 29  12 Inadimplente (0.4137931 0.5862069) *
##   3) ValorEmprestimo< 947.5 409  94 Inadimplente (0.2298289 0.7701711)  
##     6) salario>=90937.5 7   1 Adimplente (0.8571429 0.1428571) *
##     7) salario< 90937.5 402  88 Inadimplente (0.2189055 0.7810945) *
\end{verbatim}

\begin{Shaded}
\begin{Highlighting}[]
\FunctionTok{fancyRpartPlot}\NormalTok{(mytree)}
\end{Highlighting}
\end{Shaded}

\includegraphics{trabalho_2_files/figure-latex/unnamed-chunk-17-1.pdf}
Verificando acuracia da arvore de decisão

\begin{Shaded}
\begin{Highlighting}[]
\NormalTok{ds\_analise}\SpecialCharTok{$}\NormalTok{probArvore }\OtherTok{=} \FunctionTok{predict}\NormalTok{(mytree, }\AttributeTok{newdata =}\NormalTok{ ds\_analise, }\AttributeTok{type=}\StringTok{"prob"}\NormalTok{)}
\NormalTok{ds\_analise}\SpecialCharTok{$}\NormalTok{resultadoArvore }\OtherTok{=} \FunctionTok{predict}\NormalTok{(mytree, }\AttributeTok{newdata =}\NormalTok{ ds\_analise, }\AttributeTok{type=}\StringTok{"class"}\NormalTok{)}
\NormalTok{tabela\_arvore  }\OtherTok{=} \FunctionTok{table}\NormalTok{(ds\_analise}\SpecialCharTok{$}\NormalTok{default, ds\_analise}\SpecialCharTok{$}\NormalTok{resultadoArvore)}
\NormalTok{tabela\_arvore}
\end{Highlighting}
\end{Shaded}

\begin{verbatim}
##               
##                Adimplente Inadimplente
##   Adimplente           30          100
##   Inadimplente         12          331
\end{verbatim}

\begin{Shaded}
\begin{Highlighting}[]
\NormalTok{acuracia\_arvore }\OtherTok{\textless{}{-}}\NormalTok{ (tabela\_arvore[}\DecValTok{1}\NormalTok{] }\SpecialCharTok{+}\NormalTok{ tabela\_arvore[}\DecValTok{4}\NormalTok{])}\SpecialCharTok{/}\FunctionTok{sum}\NormalTok{(tabela\_arvore)}
\NormalTok{acuracia\_arvore}
\end{Highlighting}
\end{Shaded}

\begin{verbatim}
## [1] 0.7632135
\end{verbatim}

\begin{Shaded}
\begin{Highlighting}[]
\CommentTok{\# Acuracia de 76,32\%}
\end{Highlighting}
\end{Shaded}

\begin{enumerate}
\def\labelenumi{\alph{enumi})}
\setcounter{enumi}{2}
\tightlist
\item
  Verifique qual modelo teve o melhor desempenho e justifique sua
  resposta.
\end{enumerate}

A arvore de decisão teve um melhor desempenho 76,32\% de acuracia, já a
regressão logistica teve 74,41\%.

A arvore apresentou um ganho de 1,91\% de acuracia em relação à
regressão logistica

\begin{enumerate}
\def\labelenumi{\alph{enumi})}
\setcounter{enumi}{3}
\tightlist
\item
  Explique o porquê você escolheu cada uma das variáveis
\end{enumerate}

\begin{Shaded}
\begin{Highlighting}[]
\CommentTok{\# Sexo {-}\textgreater{} Para a analise poderia influenciar, se for homem ou mulher. Pensando no contexto negócio}
\CommentTok{\# Estadocivil {-}\textgreater{} Uma pessoa casada por exemplo poderia levar a uma inadimplência do que uma pessoa  solteira. Pensando no contexto negócio}
\CommentTok{\# NumerodeFilhos {-}\textgreater{} Se uma pessoa possui mais filhos a renda acaba sendo comprometida.Pensando no contexto negócio }
\CommentTok{\# TempodeResidencia {-}\textgreater{} Resolvi tirar pois não vejo que influenciaria. Pensando no contexto negócio }
\CommentTok{\# Atraso {-}\textgreater{} Retirei pois não mostrou muito significativa}
\CommentTok{\# TempodeServiço {-}\textgreater{} Exclui esta variável pois o modelo regressão logistica não estava convergindo, e                    estava muito correlacionada com o default}
\CommentTok{\# comprometido\_de\_renda {-}\textgreater{} Exclui esta variável pois o modelo regressão logistica não estava convergindo}
\CommentTok{\# faixa\_salario {-}\textgreater{} Exclui pois optei por ficar com a variável salário e tive um ganho modelo}
\CommentTok{\# Conta {-}\textgreater{} Exclui pois não estava apresentando nenhum dado (1 not defined because of singularities)}
\end{Highlighting}
\end{Shaded}


\end{document}
